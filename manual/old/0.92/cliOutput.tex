
\begin{verbatim}
usage: lusql [-A] [-a <class>] [-C <arg>] [-c <arg>] [-d <arg>] [-f <arg>] [-I <arg>] [-i <arg>] [-l <arg>] [-M <arg>]
       [-m] [-N] [-n <arg>] [-P] [-p <arg>] [-q <arg>] [-Q  <subquery>] [-Q  <subquery>] [-Q  <subquery>] [-Q  <subquery>] [-Q 
       <subquery>] [-Q  <subquery>] [-Q  <subquery>] [-Q  <subquery>] [-Q  <subquery>] [-Q  <subquery>] [-r <arg>] [-s <arg>]
       [-T] [-t] [-v]
 -A                   Append to existing Lucene index.
 -a <class>           Full name class implementing Lucene Analyzer; Default:
                      org.apache.lucene.analysis.standard.StandardAnalyzer
 -c,--jdbcUrl <arg>   JDBC connection URL: REQUIRED
 -d,--jdbc <arg>      Full name of DB driver class (should be in CLASSPATH); Default: com.mysql.jdbc.Driver
 -f <arg>             Full name class implementing DocumentFilter; Default: ca.nrc.cisti.lusql.core.NullFilter (does nothing).
                      This is applied before each Lucene Document is added to the Index. If it returns null, nothing is added
 -I <arg>             Global field index parameters. This sets all the fields parameters to this one set. Format: NNN.
                      See -i for explanatiion. Only one of -i or -I can be used at one time.
 -i <arg>             Field index parameters.
                      Format: "NNN NNN NNN".One set per field in SQL, and in same order as in SQL. Used only if you want to overide the
                      defaults (below). See for more information Field.Index, Field.Store, Field.TermVector inorg.apache.lucene.document.Field
                      http://lucene.apache.org/java/2_2_0/api/org/apache/lucene/document/Field.html
                      Default: NNN=211
                      Field Index Parameter values:
                      Index: Default:TOKENIZED
                      0:NO
                      1:NO_NORMS
                      2:TOKENIZED
                      3:UN_TOKENIZED
                      Store: Default:YES
                      0:NO
                      1:YES
                      2:COMPRESS
                      Term vector: Default:NO
                      0:NO
                      1:YES
                      2:WITH_OFFSETS
                      3:WITH_POSITIONS
                      4:WITH_POSITIONS_OFFSETS
 -l <arg>             Lucene directory. Default: index
 -M <arg>             Changes the meta replacement string for the -Qn command line parameters. Default: @
 -m                   Turns off need get around MySql driver-caused OutOfMemory problem in large queries. Sets
                      Statement.setFetchSize(Integer.MIN_VALUE)
                      See http://benjchristensen.wordpress.com/2008/05/27/mysql-jdbc-memory-usage-on-large-resultset
 -N <arg>             Number of thread for multithreading. Defaults: Runtime.getRuntime().availableProcessors()) *2.5.
                      This machine this is: 20.0
 -n <arg>             Number of documents to add. Note that this may correspond to a greater number of Db records
 -P                   Append to existing Lucene index
 -p <arg>             Properties file
 -q,--sql <arg>       SQL query (in double quotes). REQUIRED
 -Q  <subquery>       Subquery in the form "field|NNN|sql" or "field|NNN NNN...|sql" or "field|sql"
 -r <arg>             LuceneRAMBufferSizeInMBs: IndexWriter.setRAMBufferSizeMB(). Default: 48.0
 -s <arg>             Name of stop word file for Lucene to use (relative or full path)
 -T                   Turn off multithreading. Note that multithreading does not guarantee the ordering of documents. If
                      you want the order of Lucene documents to match the ordering of DB records generated by the SQL query, turn-off
                      multithreading
 -t                   Test mode. Does not open up Lucene index. Prints (-n) records from SQL query
 -v                   Verbose mode
\end{verbatim}