\begin{description}

\item [\tt -q $<$arg$>$] Primary SQL query (in double quotes). 
  This is passed directly through to the JDBC driver, so non-standard,
  DBMS specific SQL can be used.
  {\bf Required}.


\item[\tt -v] Verbose mode. This prints--out all the parameter values, as well as
  producing a dot (.) for each 1000 records processed, and a running total and
  the time for the previous 10,000 records at each 10,000 records.


\item [\tt -c $<$arg$>$] The JDBC URL to connect to the DBMS. {\bf Required}.

\item [\tt -n $<$arg$>$] Index only the first {\tt n} records from the query.


\item [\tt -l $<$arg$>$] Lucene directory. Default: {\tt ./index}. 


\item [\tt -d $<$arg$>$] Full name of DB driver class (must be in the {\tt
    CLASSPATH}). 
  Default: {\tt com.mysql.jdbc.Driver}


\item [\tt -i $<$arg$>$] LuSql allows control over how each Lucene {\tt Field} is
  indexed.
  This argument allows the user to define the 
  {\tt Field.Index}, {\tt Field.Store} and {\tt Field.TermVector}\footnote{As
    described in
    \url{http://lucene.apache.org/java/2_2_0/api/org/apache/lucene/document/Field.html}.}. 
  The format is "{\tt NNN NNN...NNN}" where each {\tt NNN} triplet corresponds
  to each field returned by the SQL query.
  The three {\tt N}s are integers and indicate the  
  {\tt Field.Index}, {\tt Field.Store} and {\tt Field.TermVector} parameters
  respectively.  
  The values for these integers are shown in Tables 
  \ref{tableIndex}, \ref{tableStore}, \ref{tableVector} in Appendix
  \ref{parameters}. 
  The defaults are {\tt Field.Index.Tokenized}, {\tt Field.Store.YES} and {\tt Field.TermVector.YES}


\item[\tt -I $<$arg$>$] Set all {\tt Field} parameters to be the same with a
  single globel {\tt NNN}. Values as per {\tt -i}.


\item [\tt -g $<$arg$>$] Define a global field=value to be included in {\bf all} {\tt
    Documents} added to the index. 
  Format: "{\tt NNN$|$fieldName=field value}"
  or "{\tt fieldName=field value}". 
  The latter uses the global {\tt -I} setting.
  Multiple {\tt -g} are allowed. {\tt NNN} values as per {\tt -i}.


\item [\tt -t] Test mode. Does not create a Lucene index. 
  Prints (-n) results records from SQL query.


\item [\tt -r $<$arg$>$] Set the RAM buffer size, in MB, of the Lucene {\tt
    IndexWriter}\footnote{\url{http://lucene.apache.org/java/2_3_1/api/org/apache/lucene/index/IndexWriter.html}{setRAMBufferSizeMB(double)}}. 
  Default: 48MB
  (different from Lucene's default of 16MB).
  A larger value will speed-up indexing, at the price of memory usage.


\item [\tt -s $<$arg$>$] Path of stop word file for Lucene to use (relative or full path).


\item [\tt -T]                   Turn off multithreading. Note that multithreading
  does not guarantee the order of documents as retrieved from the DB will
  match their order in the Lucene index.
  If for some reason you need the order of Lucene documents to match the
  order of DB records generated by the SQL query, turn off multithreading.




\item [\tt -N $<$arg$>$] Number of thread for multithreading. Default:\\
  {\tt Runtime.getRuntime().availableProcessors()} *2.5


\item[\tt -Q $<$subquery$>$] Allows for per {\tt Document} SQL subqueries to add additional information
  to the document, using a key field value from the primary query.
  Multiple {\tt -Q} can be used for multiple subqueries.
  See Section \ref{example4} for more information.
  

\item[\tt -A] Append to existing Lucene index (named in {\tt -l}), instead of
  creating anew one (default).


\item[\tt -a $<$class$>$] Full class name implementing Lucene Analyzer. \\
  Default: 
  {\tt org.apache.lucene.analysis.standard.StandardAnalyzer}.


\item[\tt -f $<$arg$>$]   Full class name implementing {\tt DocFilter}. 
  This is applied before each Lucene Document is added to the Index. If it returns null, nothing is added.
  For more information, see Section \ref{filter}.
  Default: {\tt ca.nrc.cisti.lusql.core.NullFilter} (does nothing).


\item [\tt -M $<$arg$>$] Changes the meta replacement string for the {\tt -Q}
 command line parameters. Default: {\tt }@ 


\item [\tt -m]        Turns off the need get around MySQL driver-caused
  OutOfMemory problem in large
  queries\footnote{See
    \url{http://benjchristensen.wordpress.com/2008/05/27/mysql-jdbc-memory-usage-on-large-resultset} for more information}.\\ 
  Sets
  {\tt Statement.setFetchSize(int)}\footnote{\url{http://java.sun.com/j2se/1.5.0/docs/api/java/sql/Statement.html}{setFetchSize(int)}} 
  to {\tt Integer.MIN\_VALUE}.


\item[\tt -p $<$arg$>$] Properties file




\end{description}